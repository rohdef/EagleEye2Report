\documentclass[10pt,a4paper,twocolumn]{article}
\usepackage[utf8]{inputenc}

% For dummy texts, can be deleted when we don't use those any more
\usepackage[english]{babel}
\usepackage{blindtext}

% Math packages
\usepackage{amsmath}
\usepackage{amsfonts}
\usepackage{amssymb}
\usepackage{makeidx}

% Various settings
\usepackage{graphicx}
\usepackage[left=2cm,right=2cm,top=2cm,bottom=2cm]{geometry}

% For the \maketilte command
\author{Daniel Damgaard \\
201303575 \\
daniel@damgaard.in
\and
Rohde Fischer \\
20052356 \\
rohdef@rohdef.dk
}
\title{Efficient Position Updating - Team: EagleEye}

\begin{document}
\maketitle

\tableofcontents

\section{Introduction}
\input{chapterExample}

\section{Implementation}
\input{chapterExample}

\subsection{Design choices}
\input{chapterExample}

\subsubsection{Server protocol - TCP}
\input{chapterExample}

\subsubsection{Format of stored data - JSON}
\input{chapterExample}

\subsection{User guide}
\input{chapterExample}

\section{Evaluation}
\input{chapterExample2}

\subsection{Periodic Reporting Strategy}
\input{chapterExample}

\subsection{Distance-based Reporting Strategy}
\input{chapterExample}

\subsection{Distance-based Reporting Strategy - Max Speed}
\input{chapterExample}

\subsection{Distance-based Reporting Strategy - Accelerometer}
\input{chapterExample}

\subsection{Discussion}
% Mandatory: Create screenshots using Google Earth for each of the scenarios of the collected KML files. The screenshots should include a path that marks the actually walked route. Comment on the results in the report and discuss how GPS errors impacted the results.
% Mandatory: Make a list with the following entries for each scenario: strategy, number of GPS fixes, number of uplink messages, time span, GPS fixes per second, uplink messages per second and comment on them in the report with respect to relevant literature. Discuss what pervasive positioning applications the different strategies are relevant for.
\input{chapterExample}

\section{Conclusion}
\input{chapterExample}

\end{document}
