\subsection{User guide}
The user guide is divided into the two following parts, A) Android application (Client) and B) Java application (Server).

\subsubsection*{Android application (Client)}
There is a user interface (UI) for the client, which is sketched in figure-\ref{androidgui}.

\begin{figure}[h]
\includegraphics{GUI}
\caption{Sketch of the Android GUI}
\label{androidgui}
\end{figure}

Chose one of the four algorithms, set the respective settings and press the "Start" button. Then the client sends locations to the server according to the chosen algorithm.

The server is expected to be located at rohdef.dk at port 57005, which is defined at line 37 in the class ServerRegistrar (Package: dk.au.cs.EagleEye2.registrars).

\subsubsection*{Java applikation (Server)}
There is no GUI for the server, but it takes the following command line arguments.

\begin{itemize} \itemsep1pt \parskip0pt \parsep0pt
  \item runServer: Listen for a client to send locations.
  \item parseKML: Parse the saved locations to KML.
  \item testMode: Return the text capitalized to the client.
\end{itemize}

The server listens at port 57005, which is defined at line 76 in the class Main. To run the server copy the jar-file (EagleEye2Server/out/artifacts/EagleEye2Server\_jar) to a desired location and have an folder named data in the same directory and start the server by running:

\[
\$ java -jar EagleEye2Server.jar 
\]
