\subsection{Distance-based Reporting Strategy - Max Speed}
During the analysis of the data from the distance based reporting with a maximum speed (DBRS - MS), the uncertainty of the location in meters was very high. The DBMS - MS and accelerometer (DBRS - A) is compared in figure-\ref{maxspeedaccelerometeraccuracy} and it can be seen that the accuracies are way worse than for the DBRS - A, this can also be seen on the Google Earth plot of the route.

\begin{figure}[h]
\begin{tikzpicture}
\begin{axis} [
	width=7cm,
	height=6cm,
	axis x line*=bottom,
	legend style={at={(0.5,1.2)},
	anchor=north,legend columns=-1},
	xlabel={Location number},
	axis y line*=left,
	ylabel={Accuracy (m)},
	grid=none]

\addplot[color=blue,mark=none] table {tabledata/DBRSAccelerometer.accuracies};
\addplot[color=red,mark=none] table {tabledata/DBRSMaxSpeed.accuracies};

\legend{Uncertainty (m), Max speed}
\end{axis}
\end{tikzpicture}

\caption{Comparison of uncertainties for DBRS - MS and - A. The uncertainty shows how imprecise the reading is. Measurement number is the index for the reading on the server.}
\label{maxspeedaccelerometeraccuracy}
\end{figure}

The high numbers on the uncertainty curve is probably due to the GPS switching off and having to do a cold start[REFERENCES]. This is also suggested by figure-\ref{locationsreadandsent} that shows way fewer readings by DBRS - MS. The DBRS - MS algorithm performs very well on the amount of locations read from the GPS at a rate of only $5.4$ messages per minute, but the accuracy should probably be better. It sends fixes to the server at a rate of $1.5$ per minute.